\documentclass[pep,twoside,portuguese]{iiufrgs}
\usepackage[
		pdfauthor={Matheus Pereira},
		pdftitle={Modelagem de perfil din�mico de usu�rios em comunidades de pr�tica},
		pdfstartview=FitH
	]{hyperref}	% Pacote para propriedades do documento
	
\usepackage [brazilian]{babel}		% Pacotes de linguagem
\usepackage[english]{babel}			% Pacotes de linguagem
\usepackage[T1]{fontenc}			% Pacote para conj. de caracteres correto
\usepackage[latin1]{inputenc}		% Pacote para acentua��o
%%\usepackage[ansinew]{inputenc}		% Pacote para acentua��o
%\usepackage{graphicx}				% Pacote para importar figuras
%\usepackage{color}					% Pacote para se trabalhar com cor no texto
%\usepackage{times}					% Pacote para usar fonte Adobe Times
%\usepackage{multirow}				% Pacote para utiliza��o de v�rias linhas em uma tabela
%\usepackage{amsmath}
%\usepackage{mathptmx}				% Fonte Adobe Times nas f�rmulas
%\usepackage{multicol}				% Pacote para utiliza��o de v�rias colunas em uma tabela
%\usepackage{algorithm}				% Algoritmos
%\usepackage{algorithmic}
%\usepackage{url}
%\usepackage{microtype}

\hypersetup{colorlinks,
		citecolor=black,
		filecolor=black,
		linkcolor=black,
		urlcolor=black
	}
	
%\usepackage[tight,footnotesize]{subfigure}

\title{Modelagem de perfil din�mico de \\usu�rios em comunidades de pr�tica}

\author{Matheus Pereira}

\advisor[Profa. Dra.]{Rosa Maria Vicari}


% Palavras-Chave (iniciar todas com letras min�sculas, exceto no caso de abreviaturas)
\keyword{usu�rio}
\keyword{comunidades de pr�tica}


% In�cio do Documento
\begin{document}


% Folha de Rosto
\maketitle

% Sum�rio
\tableofcontents

% Lista de Figuras
%\listoffigures

% Definir espa�amento entre par�grafos
%\setlength{\parskip}{1.5ex}

% Resumo (na l�ngua do documento)
\begin{abstract}

AAA

\end{abstract}

%%%%%%%%%%%%%%%%%%%%%%%%%%%%%%%%%%%%%%%%%%%%%%%%%%%%%%%%%%%%%%%%%%%%%%%%%%%%%%%%


\include{sections/cp1}



%%%%%%%%%%%%%%%%%%%%%%%%%%%%%%%%%%%%%%%%%%%%%%%%%%%%%%%%%%%%%%%%%%%%%%%%%%%%%%%%

\chapter{Objetivos}
\label{cha:Objetivos}

O foco principal da pesquisa
AAAAAAA \cite{jawbone}.


\section{Objetivo Geral}
\label{cha:Objetivo Geral}

Buscar compreender

\section{Objetivos Espec�ficos}
\label{cha:Objetivo Espec�ficos}

\begin{itemize}
    \item Selecionar
    \item Propor
    \item Buscar
    \item Identificar
\end{itemize}

%%%%%%%%%%%%%%%%%%%%%%%%%%%%%%%%%%%%%%%%%%%%%%%%%%%%%%%%%%%%%%%%%%%%%%%%%%%%%%%%

\chapter{Plano de Trabalho}
\label{cha:Plano}

\section{Metodologia}
\label{cha:Metodologia}

%%%%%%%%%%%%%%%%%%%%%%%%%%%%%%%%%%%%%%%%%%%%%%%%%%%%%%%%%%%%%%%%%%%%%%%%%%%%%%%%
\section{Cronograma}
\label{cha:Cronograma}

\begin{enumerate}
    \item Estudo dos trabalhos relacionados;
\end{enumerate}

%
% Refer�ncias Bibliogr�ficas
%

% \nocite{*}

%\bibliographystyle{abbrv}
\bibliographystyle{apalike}
\bibliography{references}
\end{document}
